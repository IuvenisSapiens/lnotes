\chapter{友人序}

This book was written by a friend Dr. Bao. The first version was published two years ago and this is the updated version. With many \LaTeX{} books around, is this book necessary? The answer is yes. This book is super cool. This is an advanced \LaTeX{} book and however it starts from basic and guides the readers to the most advanced tips. In addition, the style of the book is unique. It is half comments and half technical notes. If you missed the anecdotes, you miss many of the great materials.

Technical wise, the feature of the book is several fold: the presentation of digital printing technology and history, the description of software packages that can make paper/dissertation writing easier (such as how to convert a pdf to an eps file), the collection of tricks that otherwise take long time to master.

The book has 12 chapters with two Appendices. With Chapter 1 as introduction of history and Chapter 2 as introduction of \LaTeX{} as a whole from a "hello world" example, Chapter 3 s on fonts. This is a surprise arrangement and it showcases the author as an aesthete more than a software engineer. PS, \LaTeX{} lovers are more or less aesthetes, aren't they? Mathematic expressions are described in Chapter 4 and Chapter 5 is on figures. The next three chapters are not usually seen in \LaTeX{} books and they are: Metapost, PSTricks and PGF. Honestly speaking, I have not read these three chapters. Tables are mentioned in Chapter 9 and it is super useful. After reading Chapter 9, one knows how to rotate a table, how to make a long table, a wide table, etc. Chapter 10 is on structure. Chapter 11 is on printing. It gives a graphic explanation on page margins. Finally chapter 12 is on other applications such as Beamer for slides show, CV template, and letter template. The Appendices are of much use as well: Appendix A is about packages. Appendix B is on brief history of printing.

Nontechnical wise, the book is super funny with its anecdotes. The writing is in Chinese and many historical poems were re-shaped in preface. In Chapter 1, when history of digital printing was described, names and stories of computer scientists who devoted their energy in making \LaTeX{}, Adobe and etc were told. The author was obsessed with each person's background, school and degree. In footnotes, you will see sth like this 1: Ph.D. from Cambridge or PhD from MIT, etc.

The only disadvantage of this book is it is in Chinese and thus, the language limits its readers. On the other hand, with English as second language, can a book in English be that funny?

\begin{flushright}
Dr. Fan Lingling \\
August 26, 2013
\end{flushright}
