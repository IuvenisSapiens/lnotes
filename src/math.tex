\chapter{数学}

\begin{quotation}
今有上禾三秉,中禾二秉,下禾一秉,实三十九斗;上禾二秉,中禾三秉,下禾一秉,实三十四斗;上禾一秉,中禾二秉,下禾三秉,实二十六斗。问上、中、下禾实一秉各几何?
\begin{gather*}
\begin{split}
  3x+2y+z &= 39 \\
  2x+3y+z &= 34 \\
  x+2y+3z &= 26 \\
\end{split}
\end{gather*}
\begin{flushright}
--- 《九章算术》
\end{flushright}
\end{quotation}

为了使用 \AmSLaTeX 提供的数学功能,我们需要在文档的序言部分加载 \texttt{amsmath} 宏包,其详细用法可参阅其用户手册\citep{AMS_amsmath}。更全面的数学内容排版可参阅 George Grätzer\indexGratzer{} \footnote{匈牙利厄特沃什·罗兰大学 (Eötvös Loránd University) 1960年数学博士,John von Neumann 的校友。现任加拿大曼尼托巴大学 (University of Manitoba) 教授。} 的 \emph{More Math into \LaTeX}\citep{Gratzer_more_math}。

\begin{Code}[]
\usepackage{amsmath}
\end{Code}

\section{数学模式}

\LaTeX 的数学模式有两种形式:行间 (inline) 模式和独立 (display) 模式。前者是指在正文中插入数学内容;后者独立排列,可以有或没有编号。简单数学公式的输入方法见 \autoref{tab:simplemath}。

\begin{table}[htbp]
\caption{简单数学公式的输入}
\label{tab:simplemath}
\centering
\begin{tabular}{ccccc}
  \toprule
  & \TeX 命令  & \LaTeX 命令    & \LaTeX 环境    & \texttt{amsmath} 环境\\
  \midrule
  行间公式      & \verb|$...$|   & \verb|\(...\)| & \texttt{math} & \\
  无编号独立公式 & \verb|$$...$$| & \verb|\[...\]| & \texttt{displaymath} & 
    \texttt{equation*}\\
  有编号独立公式 & & & \texttt{equation} & \texttt{equation} \\
  \bottomrule
\end{tabular}
\end{table}

行间公式和无编号独立公式都有多种输入方法,新手也许会看花了眼。懒人包老师的秘诀是用最短的:行间公式用 \verb|$...$|,无编号独立公式用 \verb|\[...\]|。建议不要用 \verb|$$...$$|,因为它和 \AmSLaTeX 有冲突。\texttt{amsmath} 版本的 \texttt{equation} 环境可以嵌入次环境 (见 \ref{sec:longeq} 节) 。

\ref{sec:box} 节提到的 \verb|\fbox| 命令可以给文本内容加个方框,数学模式下也有个类似的命令 \verb|\boxed|。

\begin{example}[h]
\begin{RLDemo}[numbers=left]
Einstein's $E=mc^2$
\[ E=mc^2 \]
\[ \boxed{E=mc^2} \]
\begin{equation} 
  E=mc^2 
\end{equation}
\end{RLDemo}
\caption{数学模式}
\end{example}

\section{基本元素}

\subsection{希腊字母}

英文字母在数学模式下可以直接输入,希腊字母则需要用 \autoref{tab:greek} 中的命令输入,注意大写希腊字母的命令首字母也是大写。

\begin{table}[htbp]
\caption{希腊字母}
\label{tab:greek}
\centering
\begin{tabular}{llllllll}
  \toprule
  $\alpha$      & \verb|\alpha|      & $\theta$    & \verb|\theta|    & 
    $o$         & \verb|o|        & $\tau$     & \verb|\tau| \\
  $\beta$       & \verb|\beta|       & $\vartheta$ & \verb|\vartheta| & 
    $\pi$       & \verb|\pi|      & $\upsilon$ & \verb|\upsilon| \\
  $\gamma$      & \verb|\gamma|      & $\iota$     & \verb|\iota|     & 
    $\varpi$    & \verb|\varpi|   & $\phi$     & \verb|\phi| \\
  $\delta$      & \verb|\delta|      & $\kappa$    & \verb|\kappa|    & 
    $\rho$      & \verb|\rho|     & $\varphi$  & \verb|\varphi| \\
  $\epsilon$    & \verb|\epsilon|    & $\lambda$   & \verb|\lambda|   & 
    $\varrho$   & \verb|\varrho|  & $\chi$     & \verb|\chi| \\
  $\varepsilon$ & \verb|\varepsilon| & $\mu$       & \verb|\mu|       & 
    $\sigma$    & \verb|\sigma|   & $\psi$     & \verb|\psi| \\
  $\zeta$       & \verb|\zeta|       & $\nu$       & \verb|\nu|       & 
    $\varsigma$ & \verb|\varsigma|   & $\omega$   & \verb|\omega| \\
  $\eta$        & \verb|\eta|        & $\xi$       & \verb|\xi|       & 
    &                 &            & \\
  $\Gamma$      & \verb|\Gamma|      & $\Lambda$   & \verb|\Lambda|   & 
    $\Sigma$    & \verb|\Sigma|   & $\Psi$     & \verb|\Psi| \\
  $\Delta$      & \verb|\Delta|      & $\Xi$       & \verb|\Xi|       & 
    $\Upsilon$  & \verb|\Upsilon| & $\Omega$   & \verb|\Omega| \\
  $\Theta$      & \verb|\Theta|      & $\Pi$       & \verb|\Pi|       & 
    $\Phi$      & \verb|\Phi|     &            & \\
  \bottomrule
\end{tabular}
\end{table}

\subsection{上下标和根号}

指数或上标用 \verb|^| 表示,下标用 \verb|_| 表示,根号用 \verb|\sqrt| 表示。上下标如果多于一个字母或符号,需要用一对 \verb|{}| 括起来。

\begin{example}[h]
\begin{BTDemo}[]
\[ x_{ij}^2\quad \sqrt{x}\quad \sqrt[3]{x} \]
\end{BTDemo}
\caption{上下标和根号}
\end{example}

\subsection{分数}

分数用 \verb|\frac| 命令表示,它会根据环境自动调整字号,比如在行间公式中小一点,在独立公式中则大一点。我们可以人工设置分数字号,比如 \verb|\dfrac| 命令把分数的字号设置为独立公式中的大小,而 \verb|\tfrac| 命令则把字号设为行间公式中的大小。

\begin{example}[h]
\begin{RLDemo}[]
$ \frac{1}{2} \dfrac{1}{2} $
\[ \frac{1}{2} 
  \tfrac{1}{2} \]
\end{RLDemo}
\caption{分数}
\end{example}

\subsection{运算符}

有些小运算符例如 \verb|+ - * / =| 等可以直接输入,另一些则需要特殊命令 (见 \autoref{exa:smallop}) 。更多的数学符号可参考Pakin\indexPakin 的符号列表\citep{Pakin_comprehensive}。

\begin{example}[h]
\begin{BTDemo}[]
\[ \pm\; \times\; \div\; \cdot\; \cap\; \cup\; 
  \geq\; \leq\; \neq\; \approx\; \equiv \]
\end{BTDemo}
\caption{小运算符}
\label{exa:smallop}
\end{example}

和、积、极限、积分等大运算符用 \verb|\sum \prod \lim \int| 等命令表示 (见 \autoref{exa:bigop}) ,它们的上下标在行间公式中被压缩,以适应行高。我们也可以用 \verb|\limits| 和 \verb|\nolimits| 命令显式指定是否压缩上下标。

\begin{example}[h]
\LoadBTDemo[numbers=left]{texlet/big-ops}
\caption{大运算符}
\label{exa:bigop}
\end{example}

部分追求完美的同学可能会觉得积分公式末尾的积分变量$dx$改成$\mathrm{d}x$比较好看;另外积分函数和积分变量之间需要拉开点距离。那么我们可以用 \autoref{exa:intvar} 中的方法自己定义一个积分变量命令。

\begin{example}[h]
\begin{BTDemo}[]
\newcommand{\myd}{\;\mathrm{d}}
\[ \int x dx\quad \int x \myd x \]
\end{BTDemo}
\caption{积分变量}
\label{exa:intvar}
\end{example}

多重积分如果用多个 \verb|\int| 来输入的话,积分号之间的距离会过宽。正确的方法是用 \verb|\iint|, \verb|\iiint|, \verb|\iiiint|, \verb|\idotsint| 等命令输入。从 \autoref{exa:multint} 中我们可以看到两种方法的差异。

\begin{Code}[]
\[ \int\int\quad \int\int\int\quad 
  \int\int\int\int\quad \int\dots\int \]
\[ \iint\quad \iiint\quad \iiiint\quad \idotsint \]
\end{Code}

\begin{example}[h]
\begin{Demo}
\[ \int\int\quad \int\int\int\quad 
  \int\int\int\int\quad \int\dots\int \]
\[ \iint\quad \iiint\quad \iiiint\quad \idotsint \]
\end{Demo}
\caption{多重积分}
\label{exa:multint}
\end{example}

\subsection{箭头}

\autoref{tab:arrow} 给出了一些箭头的输入方法。\verb|\xleftarrow| 和 \verb|\xrightarrow| 命令生成的箭头可以根据内容自动调整长度(见 \autoref{exa:xarrow})。

\begin{table}[htbp]
\caption{箭头}
\label{tab:arrow}
\centering
\begin{tabular}{llll}
  \toprule
  $\leftarrow$       & \verb|\leftarrow|      & 
    $\longleftarrow$       & \verb|\longleftarrow| \\
  $\rightarrow$      & \verb|\rightarrow|     & 
    $\longrightarrow$      & \verb|\longrightarrow| \\
  $\leftrightarrow$  & \verb|\leftrightarrow| & 
    $\longleftrightarrow$  & \verb|\longleftrightarrow| \\
  $\Leftarrow$       & \verb|\Leftarrow|      & 
    $\Longleftarrow$       & \verb|\Longleftarrow| \\
  $\Rightarrow$      & \verb|\Rightarrow|     & 
    $\Longrightarrow$      & \verb|\Longrightarrow| \\
  $\Leftrightarrow$  & \verb|\Leftrightarrow| & 
    $\Longleftrightarrow$  & \verb|\Longleftrightarrow| \\
  \bottomrule
\end{tabular}
\end{table}

\begin{example}[h]
\begin{RLDemo}[]
\[ \xleftarrow{x+y+z}\quad
\xrightarrow[x<y]{a*b*c} \]
\end{RLDemo}
\caption{可扩展箭头}
\label{exa:xarrow}
\end{example}

\subsection{注音和标注}

\autoref{tab:mathaccent} 列出一些数学注音符号 (accent) ,\autoref{tab:notation} 列出一些长的标注符号。

\begin{table}[!htbp]
\caption{数学注音符号}
\label{tab:mathaccent}
\centering
\begin{tabular}{llllllll}
  \toprule
  $\bar{x}$  & \verb|\bar{x}| & $\acute{x}$ & \verb|\acute{x}| &
    $\mathring{x}$ & \verb|\mathring{x}| \\
  $\vec{x}$ & \verb|\vec{x}| & $\grave{x}$ & \verb|\grave{x}| &
    $\dot{x}$  & \verb|\dot{x}| \\
  $\hat{x}$ & \verb|\hat{x}| & $\tilde{x}$ & \verb|\tilde{x}| &
    $\ddot{x}$  & \verb|\ddot{x}| \\
  $\check{x}$ & \verb|\check{x}| & $\breve{x}$ & \verb|\breve{x}| & 
    $\dddot{x}$ & \verb|\dddot{x}| \\
  \bottomrule
\end{tabular}
\end{table}

\begin{table}[htbp]
\caption{长标注符号}
\label{tab:notation}
\centering
\begin{tabular}{llll}
  \toprule
  $\overline{xxx}$        & \verb|\overline{xxx}|        & 
    $\overleftrightarrow{xxx}$ & \verb|\overleftrightarrow{xxx}| \\
  $\underline{xxx}$       & \verb|\underline{xxx}|       & 
    $\underleftrightarrow{xxx}$ & \verb|\underleftrightarrow{xxx}| \\
  $\overleftarrow{xxx}$   & \verb|\overleftarrow{xxx}|   & 
    $\overbrace{xxx}$   & \verb|\overbrace{xxx}| \\
  $\underleftarrow{xxx}$  & \verb|\underleftarrow{xxx}|  & 
    $\underbrace{xxx}$  & \verb|\underbrace{xxx}| \\
  $\overrightarrow{xxx}$  & \verb|\overrightarrow{xxx}|  & 
    $\widehat{xxx}$     & \verb|\widehat{xxx}| \\
  $\underrightarrow{xxx}$ & \verb|\underrightarrow{xxx}| & 
    $\widetilde{xxx}$   & \verb|\widetilde{xxx}| \\
  \bottomrule
\end{tabular}
\end{table}

\subsection{分隔符}

各种括号用 \verb|() [] \{\} \langle\rangle| 等符号或命令表示;花括号通常用来输入命令和环境的参数,所以在数学公式中它们前面要加 \verb|\|。因为 \LaTeX 中 \verb+|+ 和 \verb+\|+ 的应用过于随意,\texttt{amsmath} 宏包推荐用 \verb|\lvert\rvert| 和 \verb|\lVert\rVert| 取而代之。

我们可以在上述分隔符前面加 \verb|\big \Big \bigg \Bigg| 等命令来调整其大小。\LaTeX 原有的方法是在分隔符前面加 \verb|\left \right| 来自动调整大小,但是效果不佳,所以\texttt{amsmath} 不推荐用这种方法。

\begin{example}[h]
\LoadBTDemo[numbers=left]{texlet/delimiters}
\caption{分隔符}
\label{exa:delimiters}
\end{example}

\subsection{省略号}

省略号用 \verb|\dots \cdots \vdots \ddots| 等命令表示。\verb|\dots| 和 \verb|\cdots| 的纵向位置不同;前者一般用于有下标的序列。

\begin{Code}[]
\[ x_1,x_2,\dots,x_n\quad 1,2,\cdots,n\quad 
  \vdots\quad \ddots \]
\end{Code}

\begin{example}[h]
\begin{Demo}
\[ x_1,x_2,\dots,x_n\quad 1,2,\cdots,n\quad 
  \vdots\quad \ddots \]
\end{Demo}
\caption{省略号}
\label{exa:dots}
\end{example}

\subsection{空白间距}

在数学模式中,我们可以用 \autoref{tab:quad} 中的命令生成合适的空白间距,注意负间距命令 \verb|\!| 可以用来减小间距。

\begin{table}[htbp]
\caption{空白间距}
\label{tab:quad}
\centering
\begin{tabular}{llllll}
  \toprule
  \verb|\,| & 3/18em & $|\,|$ & \verb|\quad| & 1em & $|\quad|$ \\
  \verb|\:| & 4/18em & $|\:|$ & \verb|\qquad| & 2em & $|\qquad|$ \\
  \verb|\;| & 5/18em & $|\;|$ & \verb|\!| & -3/18em & $|\!|$ \\
  \bottomrule
\end{tabular}
\end{table}

\section{矩阵}

数学模式下可以用 \texttt{array} 环境(见 \autoref{exa:array})来生成矩阵,它提供了外部对齐和列对齐的控制参数。外部对齐是指整个矩阵和周围对象的纵向关系,有三种方式:居顶、居中 (缺省) 、居底,分别用 \texttt{t}, \texttt{c}, \texttt{b} 来表示;列对齐也有三种方式:居左、居中、居右,分别用 \texttt{l}, \texttt{c}, \texttt{r} 表示。\verb|\\| 和 \verb|&| 用来分隔行和列。

其语法如下:

\begin{Code}[]
\begin{array}[`外部对齐`]{`列对齐`}
  `行列内容`
\end{array}
\end{Code}

\begin{example}[htbp]
\begin{RLDemo}[numbers=left]
\[ \begin{array}{ccc}
x_1 & x_2 & \dots \\
x_3 & x_4 & \dots \\
\vdots & \vdots & \ddots
\end{array} \]
\end{RLDemo}
\caption{矩阵}
\label{exa:array}
\end{example}

\texttt{amsmath} 的 \texttt{pmatrix}, \texttt{bmatrix}, \texttt{Bmatrix}, \texttt{vmatrix}, \texttt{Vmatrix} 等环境可以在矩阵两边加上各种分隔符,但是它们没有对齐方式参数(见 \autoref{exa:matrix})。

\begin{example}[htbp]
\LoadBTDemo[numbers=left]{texlet/matrix}
\caption{更多矩阵}
\label{exa:matrix}
\end{example}

\verb|\smallmatrix| 命令可以生成行间矩阵(见 \autoref{exa:smallmatrix})。

\begin{example}[!h]
\begin{BTDemo}[]
Marry has a little matrix $ ( \begin{smallmatrix} 
a&b\\c&d \end{smallmatrix} ) $.
\end{BTDemo}
\caption{行间矩阵}
\label{exa:smallmatrix}
\end{example}

\section{多行公式}

有时一个公式太长一行放不下,或几个公式需要写成一组,这时我们就要用到\texttt{amsmath} 提供的一些多行公式环境。

\subsection{长公式}
\label{sec:longeq}

无须对齐的长公式可以使用 \texttt{multline} 环境(见 \autoref{exa:multline})。需要对齐的长公式可以使用 \texttt{split} 环境(见 \autoref{exa:split}),它本身不能独立使用,必须包含在其它数学环境内,因此也称作次环境。它用 \verb|\\| 和 \verb|&| 来分行和设置对齐的位置。

\begin{example}[htbp]
\begin{RLDemo}[]
\begin{multline}
x = a+b+c+{} \\
  d+e+f+g
\end{multline}
\end{RLDemo}
\caption{无对齐长公式}
\label{exa:multline}
\end{example}

\begin{example}[htbp]
\begin{RLDemo}[]
\[ \begin{split}
x ={} &a+b+c+{} \\
      &d+e+f+g
\end{split} \]
\end{RLDemo}
\caption{对齐长公式}
\label{exa:split}
\end{example}

\subsection{公式组}

不需要对齐的公式组可以使用 \texttt{gather} 环境(见 \autoref{exa:gather}),需要对齐的公式组用 \texttt{align} 环境(见 \autoref{exa:align})。

\begin{example}[htbp]
\begin{RLDemo}[]
\begin{gather}
a = b+c+d \\
x = y+z
\end{gather}
\end{RLDemo}
\caption{无对齐公式组}
\label{exa:gather}
\end{example}

\begin{example}[!h]
\begin{RLDemo}[]
\begin{align}
a &= b+c+d \\
x &= y+z
\end{align}
\end{RLDemo}
\caption{对齐公式组}
\label{exa:align}
\end{example}

\texttt{multline}, \texttt{gather}, \texttt{align} 等环境都有带 \texttt{*} 的版本,不生成公式编号。

\subsection{分支公式}

分段函数通常用 \texttt{cases} 次环境写成分支公式(见 \autoref{exa:cases})。

\begin{example}[htbp]
\begin{RLDemo}[]
\[ y=\begin{cases}
  -x,\quad x\leq 0 \\
  x,\quad x>0
\end{cases} \]
\end{RLDemo}
\caption{分支公式}
\label{exa:cases}
\end{example}

\section{定理和证明}

\verb|\newtheorem| 命令可以用来定义定理之类的环境,其语法如下。

\verb|语法:{环境名}[编号延续]{显示名}[编号层次]|

下面的代码定制了四个环境:定义、定理、引理和推论,它们都在一个 \texttt{section} 内统一编号,而引理和推论会延续定理的编号。我们在 \autoref{exa:def_theorem} 中定制了一些环境后,可以像 \autoref{exa:use_theorem} 那样使用它们。

\begin{example}[htbp]
\begin{Code}[]
\newtheorem{definition}{`定义`}[section]
\newtheorem{theorem}{`定理`}[section]
\newtheorem{lemma}[theorem]{`引理`}
\newtheorem{corollary}[theorem]{`推论`}
\end{Code}
\caption{定制定理类环境}
\label{exa:def_theorem}
\end{example}

\newtheorem{definition}{定义}[section]
\newtheorem{theorem}{定理}[section]
\newtheorem{lemma}[theorem]{引理}
\newtheorem{corollary}[theorem]{推论}

\begin{example}[h]
\LoadFRLDemo[]{texlet/theorem-def}
\LoadFRLDemo[]{texlet/theorem-the}
\LoadFRLDemo[]{texlet/theorem-lem}
\LoadFRLDemo[]{texlet/theorem-cor}
\caption{使用定理类环境}
\label{exa:use_theorem}
\end{example}

\texttt{amsthm} 宏包提供的 \texttt{proof} 环境(见 \autoref{exa:proof})可以用来输入证明,它会在证明结尾加一个QED符号 \footnote{拉丁语quod erat demonstrandum的缩写。}。

\LoadCode[]{texlet/theorem-proof-esc}

\begin{example}[htbp]
\LoadDemo{texlet/theorem-proof}
\caption{证明}
\label{exa:proof}
\end{example}

\section{数学字体}

和文本模式类似,我们在数学模式下也可以选用不同的字体样式 (见 \autoref{tab:mathfont}) 
。\verb|\mathbb| 和 \verb|\mathfrak| 需要 \texttt{amsfonts} 宏包,\verb|\mathscr| 需要 \texttt{mathrsfs} 宏包。

\begin{table}[htbp]
\caption{数学字体}
\label{tab:mathfont}
\centering
\begin{tabular}{llll}
  \toprule
  缺省 & $ABCXYZ$ & 
  \verb|\mathbf| & $\mathbf{ABCXYZ}$ \\
  \verb|\mathrm| & $\mathrm{ABCXYZ}$ & 
  \verb|\mathit| & $\mathit{ABCXYZ}$ \\
  \verb|\mathsf| & $\mathsf{ABCXYZ}$ & 
  \verb|\mathbb| & $\mathbb{ABCXYZ}$ \\
  \verb|\mathtt| & $\mathtt{ABCXYZ}$ & 
  \verb|\mathfrak| & $\mathfrak{ABCXYZ}$ \\
  \verb|\mathcal| & $\mathcal{ABCXYZ}$ & 
  \verb|\mathscr| & $\mathscr{ABCXYZ}$ \\
  \bottomrule
\end{tabular}
\end{table}

\bibliographystyle{unsrtnat}
\bibliography{lnotes2}
